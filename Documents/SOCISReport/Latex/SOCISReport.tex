%%%%%%%%%%%%%%%%%%%%%%%%%%%%%%%%%%%%%%%%%%%%%%%%%%%%%%%%%%%%%%%%%%%%%
%
% Complete documentation on the extended LaTeX markup used for Insight
% documentation is available in ``Documenting Insight'', that is part
% of the standard documentation for Insight.  It may be found online
% at:
%
%                    http://www.itk.org
%
%%%%%%%%%%%%%%%%%%%%%%%%%%%%%%%%%%%%%%%%%%%%%%%%%%%%%%%%%%%%%%%%%%%%%

\documentclass{InsightSoftwareGuide}

\usepackage[pdftex]{graphicx}

\usepackage{times,lscape,url}

\usepackage[latin1]{inputenc}

\usepackage{tikz}

\usepackage{color}

\usepackage{xspace}

\definecolor{listcomment}{rgb}{0.0,0.5,0.0}
\definecolor{listkeyword}{rgb}{0.0,0.0,0.5}
\definecolor{listnumbers}{gray}{0.65}
\definecolor{listlightgray}{gray}{0.955}
\definecolor{listwhite}{gray}{1.0}

\usepackage{listings}
\newcommand{\lstsetcpp}
{
\lstset{frame = tb,
       framerule = 0.25pt,
       float,
       fontadjust,
       backgroundcolor={\color{listlightgray}},
       basicstyle = {\ttfamily\footnotesize},
       keywordstyle = {\ttfamily\color{listkeyword}\textbf},
       identifierstyle = {\ttfamily},
       commentstyle = {\ttfamily\color{listcomment}\textit},
       stringstyle = {\ttfamily},
       showstringspaces = false,
       showtabs = false,
       numbers = none,
       numbersep = 6pt,
       numberstyle={\ttfamily\color{listnumbers}},
       tabsize = 2,
       language=[ANSI]C++,
       floatplacement=!h
       }
}
\newcommand{\lstsetpython}
{
\lstset{language=Python
        }
}
\newcommand{\lstsetjava}
{
\lstset{language=Java
        }
}


\newif\ifitkFullVersion
\itkFullVersiontrue
%\itkFullVersionfalse

\newif\ifitkPrintedVersion
\itkPrintedVersiontrue
%\itkPrintedVersionfalse


%%%%%%%%%%%%%%%%%%%%%%%%%%%%%%%%%%%%%%%%%%%%%%%%%%%%%%%%%%%%%%%%%%
%
%  hyperref should be the last package to be loaded.
%
%%%%%%%%%%%%%%%%%%%%%%%%%%%%%%%%%%%%%%%%%%%%%%%%%%%%%%%%%%%%%%%%%%
\usepackage[pdftex,
pdftitle={The Orfeo ToolBox Cookbook, a guide for non-developers},
pdfauthor={CNES},
pdfsubject={Remote Sensing, Orfeo, Pleiades, Cosmo Skymed},
pdfkeywords={image processing, Remote sensing, Guide},
pdfpagemode={UseOutlines},
bookmarks,bookmarksopen,
pdfstartview={FitH},
backref,
colorlinks,linkcolor={red},citecolor={blue},urlcolor={blue},
]{hyperref}

\usepackage{amsmath,amssymb,amsfonts}
\usepackage{bbm}

\def\logoCNES{../Art/logoVectoriel.pdf}



% Useful macros
\newcommand{\otb}{\textbf{Orfeo ToolBox}\xspace}
\newcommand{\app}{\textbf{OTB-Applications}\xspace}
\newcommand{\mont}{\textbf{Monteverdi}\xspace}
\newcommand{\mmod}[1]{\emph{#1}\index{#1}\xspace}
\newcommand{\application}[1]{\emph{#1}\index{#1}\xspace}
\newcommand{\ossim}{\href{http://www.ossim.org/OSSIM/OSSIM_Home.html}{OSSIM}\xspace}
\newcommand{\sg}{\href{http://orfeo-toolbox.org/SoftwareGuide/}{OTB Software Guide}\xspace}
\newcommand{\dox}{\href{http://orfeo-toolbox.org/doxygen/}{Doxygen}\xspace}
\newcommand{\website}{\href{http://orfeo-toolbox.org}{Orfeo ToolBox website}\xspace}
\newcommand{\qgis}{\href{http://www.qgis.org/}{Quantum GIS}\xspace}
\newcommand{\gdal}{\href{http://www.gdal.org/}{GDAL}\xspace}
\newcommand{\osgeow}{\href{http://trac.osgeo.org/osgeo4w/}{OSGeo4W}\xspace}
\newcommand{\download}{\href{http://sourceforge.net/projects/orfeo-toolbox/}{OTB download
   page}\xspace}
\newcommand{\googleearth}{\textbf{Google Earth\copyright}\xspace}

\newtheorem{algo}{Algorithm}
\newtheorem{defin}{Definition}

\title{SOCIS 2011 : InSAR processing \\with OTB library}

\author{SOCIS Team}

\authoraddress{
  \url{http://www.orfeo-toolbox.org}\\
  e-mail: \email{otb@cnes.fr}
}

\date{\today}


% actually write the .idx file
\makeindex

\setcounter{tocdepth}{3}

%%%%%%%%%%%%%%%%%%%%%%%%%%%%%%%%%%%%%%%%%%%%%%%%%%%%%%%%%%%%%%%%%%%
%
%           Begin Document
%
%%%%%%%%%%%%%%%%%%%%%%%%%%%%%%%%%%%%%%%%%%%%%%%%%%%%%%%%%%%%%%%%%%%

\begin{document}

\ifitkPrintedVersion
\fi

\maketitle

\frontmatter

\hyperbaseurl{http://www.orfeo-toolbox.org}

\lstsetcpp


%%%%%%%%%%%%%%%%%%%%%%%%%%%%%%%%%%%%%%%%%%%%%%
%
% remove headings from the following material
\pagestyle{plain}
%
%%%%%%%%%%%%%%%%%%%%%%%%%%%%%%%%%%%%%%%%%%%%%%

%%\ifitkPrintedVersion
%% \input{Cover.tex}
%%\fi

%\input{Abstract.tex}
\chapter*{Foreword}
\noindent

ESA Summer of Code in Space (SOCIS) is a pilot program run 
by the Advanced Concepts Team of the European Space Agency 
that offers student developers stipends to write code for various 
space-related open source software projects. 
Through SOCIS, accepted student applicants are paired with a mentor 
or mentors from the participating projects, thus gaining exposure to 
real-world software development scenarios. In turn, the participating 
projects are able to more easily identify and bring in new developers.

InSAR framework has been choose.  

After almost 5 years of development, the \otb has become a
rich library used in many remote sensing context, from research work
to operational systems.  For more information on the \otb, please 
feel free to visit the \website.

%%%%%%%%%%%%%%%%%%%%%%%%%%%%%%%%%%%%%%%%%%%%%%%%%%%%%%%%%
%
% Insert Table of Contents; List of Figures and Tables
%
%%%%%%%%%%%%%%%%%%%%%%%%%%%%%%%%%%%%%%%%%%%%%%%%%%%%%%%%%


%%%%%%%%%%%%%%%%%%%%%%%%%%%%%%%%%%%%%%%%%%%%%%
%
% enable headings from the following material
\pagestyle{normal}
%
%%%%%%%%%%%%%%%%%%%%%%%%%%%%%%%%%%%%%%%%%%%%%%
\small
\tableofcontents
\listoffigures
\listoftables
\normalsize

%%%%%%%%%%%%%%%%%%%%%%%%%%%%%%%%%%%%%%%%%
%
% Begin technical content
%
%%%%%%%%%%%%%%%%%%%%%%%%%%%%%%%%%%%%%%%%%

\mainmatter

\chapter{InSAR Processing}\label{chap:insar-processing}

\section{Introduction}\label{sec:insarintro}





\end{document}



